\documentclass[11pt]{article}

    \usepackage[breakable]{tcolorbox}
    \usepackage{parskip} % Stop auto-indenting (to mimic markdown behaviour)
    

    % Basic figure setup, for now with no caption control since it's done
    % automatically by Pandoc (which extracts ![](path) syntax from Markdown).
    \usepackage{graphicx}
    % Maintain compatibility with old templates. Remove in nbconvert 6.0
    \let\Oldincludegraphics\includegraphics
    % Ensure that by default, figures have no caption (until we provide a
    % proper Figure object with a Caption API and a way to capture that
    % in the conversion process - todo).
    \usepackage{caption}
    \DeclareCaptionFormat{nocaption}{}
    \captionsetup{format=nocaption,aboveskip=0pt,belowskip=0pt}

    \usepackage{float}
    \floatplacement{figure}{H} % forces figures to be placed at the correct location
    \usepackage{xcolor} % Allow colors to be defined
    \usepackage{enumerate} % Needed for markdown enumerations to work
    \usepackage{geometry} % Used to adjust the document margins
    \usepackage{amsmath} % Equations
    \usepackage{amssymb} % Equations
    \usepackage{textcomp} % defines textquotesingle
    % Hack from http://tex.stackexchange.com/a/47451/13684:
    \AtBeginDocument{%
        \def\PYZsq{\textquotesingle}% Upright quotes in Pygmentized code
    }
    \usepackage{upquote} % Upright quotes for verbatim code
    \usepackage{eurosym} % defines \euro

    \usepackage{iftex}
    \ifPDFTeX
        \usepackage[T1]{fontenc}
        \IfFileExists{alphabeta.sty}{
              \usepackage{alphabeta}
          }{
              \usepackage[mathletters]{ucs}
              \usepackage[utf8x]{inputenc}
          }
    \else
        \usepackage{fontspec}
        \usepackage{unicode-math}
    \fi

    \usepackage{fancyvrb} % verbatim replacement that allows latex
    \usepackage{grffile} % extends the file name processing of package graphics
                         % to support a larger range
    \makeatletter % fix for old versions of grffile with XeLaTeX
    \@ifpackagelater{grffile}{2019/11/01}
    {
      % Do nothing on new versions
    }
    {
      \def\Gread@@xetex#1{%
        \IfFileExists{"\Gin@base".bb}%
        {\Gread@eps{\Gin@base.bb}}%
        {\Gread@@xetex@aux#1}%
      }
    }
    \makeatother
    \usepackage[Export]{adjustbox} % Used to constrain images to a maximum size
    \adjustboxset{max size={0.9\linewidth}{0.9\paperheight}}

    % The hyperref package gives us a pdf with properly built
    % internal navigation ('pdf bookmarks' for the table of contents,
    % internal cross-reference links, web links for URLs, etc.)
    \usepackage{hyperref}
    % The default LaTeX title has an obnoxious amount of whitespace. By default,
    % titling removes some of it. It also provides customization options.
    \usepackage{titling}
    \usepackage{longtable} % longtable support required by pandoc >1.10
    \usepackage{booktabs}  % table support for pandoc > 1.12.2
    \usepackage{array}     % table support for pandoc >= 2.11.3
    \usepackage{calc}      % table minipage width calculation for pandoc >= 2.11.1
    \usepackage[inline]{enumitem} % IRkernel/repr support (it uses the enumerate* environment)
    \usepackage[normalem]{ulem} % ulem is needed to support strikethroughs (\sout)
                                % normalem makes italics be italics, not underlines
    \usepackage{mathrsfs}
    

    
    % Colors for the hyperref package
    \definecolor{urlcolor}{rgb}{0,.145,.698}
    \definecolor{linkcolor}{rgb}{.71,0.21,0.01}
    \definecolor{citecolor}{rgb}{.12,.54,.11}

    % ANSI colors
    \definecolor{ansi-black}{HTML}{3E424D}
    \definecolor{ansi-black-intense}{HTML}{282C36}
    \definecolor{ansi-red}{HTML}{E75C58}
    \definecolor{ansi-red-intense}{HTML}{B22B31}
    \definecolor{ansi-green}{HTML}{00A250}
    \definecolor{ansi-green-intense}{HTML}{007427}
    \definecolor{ansi-yellow}{HTML}{DDB62B}
    \definecolor{ansi-yellow-intense}{HTML}{B27D12}
    \definecolor{ansi-blue}{HTML}{208FFB}
    \definecolor{ansi-blue-intense}{HTML}{0065CA}
    \definecolor{ansi-magenta}{HTML}{D160C4}
    \definecolor{ansi-magenta-intense}{HTML}{A03196}
    \definecolor{ansi-cyan}{HTML}{60C6C8}
    \definecolor{ansi-cyan-intense}{HTML}{258F8F}
    \definecolor{ansi-white}{HTML}{C5C1B4}
    \definecolor{ansi-white-intense}{HTML}{A1A6B2}
    \definecolor{ansi-default-inverse-fg}{HTML}{FFFFFF}
    \definecolor{ansi-default-inverse-bg}{HTML}{000000}

    % common color for the border for error outputs.
    \definecolor{outerrorbackground}{HTML}{FFDFDF}

    % commands and environments needed by pandoc snippets
    % extracted from the output of `pandoc -s`
    \providecommand{\tightlist}{%
      \setlength{\itemsep}{0pt}\setlength{\parskip}{0pt}}
    \DefineVerbatimEnvironment{Highlighting}{Verbatim}{commandchars=\\\{\}}
    % Add ',fontsize=\small' for more characters per line
    \newenvironment{Shaded}{}{}
    \newcommand{\KeywordTok}[1]{\textcolor[rgb]{0.00,0.44,0.13}{\textbf{{#1}}}}
    \newcommand{\DataTypeTok}[1]{\textcolor[rgb]{0.56,0.13,0.00}{{#1}}}
    \newcommand{\DecValTok}[1]{\textcolor[rgb]{0.25,0.63,0.44}{{#1}}}
    \newcommand{\BaseNTok}[1]{\textcolor[rgb]{0.25,0.63,0.44}{{#1}}}
    \newcommand{\FloatTok}[1]{\textcolor[rgb]{0.25,0.63,0.44}{{#1}}}
    \newcommand{\CharTok}[1]{\textcolor[rgb]{0.25,0.44,0.63}{{#1}}}
    \newcommand{\StringTok}[1]{\textcolor[rgb]{0.25,0.44,0.63}{{#1}}}
    \newcommand{\CommentTok}[1]{\textcolor[rgb]{0.38,0.63,0.69}{\textit{{#1}}}}
    \newcommand{\OtherTok}[1]{\textcolor[rgb]{0.00,0.44,0.13}{{#1}}}
    \newcommand{\AlertTok}[1]{\textcolor[rgb]{1.00,0.00,0.00}{\textbf{{#1}}}}
    \newcommand{\FunctionTok}[1]{\textcolor[rgb]{0.02,0.16,0.49}{{#1}}}
    \newcommand{\RegionMarkerTok}[1]{{#1}}
    \newcommand{\ErrorTok}[1]{\textcolor[rgb]{1.00,0.00,0.00}{\textbf{{#1}}}}
    \newcommand{\NormalTok}[1]{{#1}}

    % Additional commands for more recent versions of Pandoc
    \newcommand{\ConstantTok}[1]{\textcolor[rgb]{0.53,0.00,0.00}{{#1}}}
    \newcommand{\SpecialCharTok}[1]{\textcolor[rgb]{0.25,0.44,0.63}{{#1}}}
    \newcommand{\VerbatimStringTok}[1]{\textcolor[rgb]{0.25,0.44,0.63}{{#1}}}
    \newcommand{\SpecialStringTok}[1]{\textcolor[rgb]{0.73,0.40,0.53}{{#1}}}
    \newcommand{\ImportTok}[1]{{#1}}
    \newcommand{\DocumentationTok}[1]{\textcolor[rgb]{0.73,0.13,0.13}{\textit{{#1}}}}
    \newcommand{\AnnotationTok}[1]{\textcolor[rgb]{0.38,0.63,0.69}{\textbf{\textit{{#1}}}}}
    \newcommand{\CommentVarTok}[1]{\textcolor[rgb]{0.38,0.63,0.69}{\textbf{\textit{{#1}}}}}
    \newcommand{\VariableTok}[1]{\textcolor[rgb]{0.10,0.09,0.49}{{#1}}}
    \newcommand{\ControlFlowTok}[1]{\textcolor[rgb]{0.00,0.44,0.13}{\textbf{{#1}}}}
    \newcommand{\OperatorTok}[1]{\textcolor[rgb]{0.40,0.40,0.40}{{#1}}}
    \newcommand{\BuiltInTok}[1]{{#1}}
    \newcommand{\ExtensionTok}[1]{{#1}}
    \newcommand{\PreprocessorTok}[1]{\textcolor[rgb]{0.74,0.48,0.00}{{#1}}}
    \newcommand{\AttributeTok}[1]{\textcolor[rgb]{0.49,0.56,0.16}{{#1}}}
    \newcommand{\InformationTok}[1]{\textcolor[rgb]{0.38,0.63,0.69}{\textbf{\textit{{#1}}}}}
    \newcommand{\WarningTok}[1]{\textcolor[rgb]{0.38,0.63,0.69}{\textbf{\textit{{#1}}}}}


    % Define a nice break command that doesn't care if a line doesn't already
    % exist.
    \def\br{\hspace*{\fill} \\* }
    % Math Jax compatibility definitions
    \def\gt{>}
    \def\lt{<}
    \let\Oldtex\TeX
    \let\Oldlatex\LaTeX
    \renewcommand{\TeX}{\textrm{\Oldtex}}
    \renewcommand{\LaTeX}{\textrm{\Oldlatex}}
    % Document parameters
    % Document title
    \title{notebooks\_TensorFlow\_Object\_Detection\_API}
    
    
    
    
    
% Pygments definitions
\makeatletter
\def\PY@reset{\let\PY@it=\relax \let\PY@bf=\relax%
    \let\PY@ul=\relax \let\PY@tc=\relax%
    \let\PY@bc=\relax \let\PY@ff=\relax}
\def\PY@tok#1{\csname PY@tok@#1\endcsname}
\def\PY@toks#1+{\ifx\relax#1\empty\else%
    \PY@tok{#1}\expandafter\PY@toks\fi}
\def\PY@do#1{\PY@bc{\PY@tc{\PY@ul{%
    \PY@it{\PY@bf{\PY@ff{#1}}}}}}}
\def\PY#1#2{\PY@reset\PY@toks#1+\relax+\PY@do{#2}}

\@namedef{PY@tok@w}{\def\PY@tc##1{\textcolor[rgb]{0.73,0.73,0.73}{##1}}}
\@namedef{PY@tok@c}{\let\PY@it=\textit\def\PY@tc##1{\textcolor[rgb]{0.24,0.48,0.48}{##1}}}
\@namedef{PY@tok@cp}{\def\PY@tc##1{\textcolor[rgb]{0.61,0.40,0.00}{##1}}}
\@namedef{PY@tok@k}{\let\PY@bf=\textbf\def\PY@tc##1{\textcolor[rgb]{0.00,0.50,0.00}{##1}}}
\@namedef{PY@tok@kp}{\def\PY@tc##1{\textcolor[rgb]{0.00,0.50,0.00}{##1}}}
\@namedef{PY@tok@kt}{\def\PY@tc##1{\textcolor[rgb]{0.69,0.00,0.25}{##1}}}
\@namedef{PY@tok@o}{\def\PY@tc##1{\textcolor[rgb]{0.40,0.40,0.40}{##1}}}
\@namedef{PY@tok@ow}{\let\PY@bf=\textbf\def\PY@tc##1{\textcolor[rgb]{0.67,0.13,1.00}{##1}}}
\@namedef{PY@tok@nb}{\def\PY@tc##1{\textcolor[rgb]{0.00,0.50,0.00}{##1}}}
\@namedef{PY@tok@nf}{\def\PY@tc##1{\textcolor[rgb]{0.00,0.00,1.00}{##1}}}
\@namedef{PY@tok@nc}{\let\PY@bf=\textbf\def\PY@tc##1{\textcolor[rgb]{0.00,0.00,1.00}{##1}}}
\@namedef{PY@tok@nn}{\let\PY@bf=\textbf\def\PY@tc##1{\textcolor[rgb]{0.00,0.00,1.00}{##1}}}
\@namedef{PY@tok@ne}{\let\PY@bf=\textbf\def\PY@tc##1{\textcolor[rgb]{0.80,0.25,0.22}{##1}}}
\@namedef{PY@tok@nv}{\def\PY@tc##1{\textcolor[rgb]{0.10,0.09,0.49}{##1}}}
\@namedef{PY@tok@no}{\def\PY@tc##1{\textcolor[rgb]{0.53,0.00,0.00}{##1}}}
\@namedef{PY@tok@nl}{\def\PY@tc##1{\textcolor[rgb]{0.46,0.46,0.00}{##1}}}
\@namedef{PY@tok@ni}{\let\PY@bf=\textbf\def\PY@tc##1{\textcolor[rgb]{0.44,0.44,0.44}{##1}}}
\@namedef{PY@tok@na}{\def\PY@tc##1{\textcolor[rgb]{0.41,0.47,0.13}{##1}}}
\@namedef{PY@tok@nt}{\let\PY@bf=\textbf\def\PY@tc##1{\textcolor[rgb]{0.00,0.50,0.00}{##1}}}
\@namedef{PY@tok@nd}{\def\PY@tc##1{\textcolor[rgb]{0.67,0.13,1.00}{##1}}}
\@namedef{PY@tok@s}{\def\PY@tc##1{\textcolor[rgb]{0.73,0.13,0.13}{##1}}}
\@namedef{PY@tok@sd}{\let\PY@it=\textit\def\PY@tc##1{\textcolor[rgb]{0.73,0.13,0.13}{##1}}}
\@namedef{PY@tok@si}{\let\PY@bf=\textbf\def\PY@tc##1{\textcolor[rgb]{0.64,0.35,0.47}{##1}}}
\@namedef{PY@tok@se}{\let\PY@bf=\textbf\def\PY@tc##1{\textcolor[rgb]{0.67,0.36,0.12}{##1}}}
\@namedef{PY@tok@sr}{\def\PY@tc##1{\textcolor[rgb]{0.64,0.35,0.47}{##1}}}
\@namedef{PY@tok@ss}{\def\PY@tc##1{\textcolor[rgb]{0.10,0.09,0.49}{##1}}}
\@namedef{PY@tok@sx}{\def\PY@tc##1{\textcolor[rgb]{0.00,0.50,0.00}{##1}}}
\@namedef{PY@tok@m}{\def\PY@tc##1{\textcolor[rgb]{0.40,0.40,0.40}{##1}}}
\@namedef{PY@tok@gh}{\let\PY@bf=\textbf\def\PY@tc##1{\textcolor[rgb]{0.00,0.00,0.50}{##1}}}
\@namedef{PY@tok@gu}{\let\PY@bf=\textbf\def\PY@tc##1{\textcolor[rgb]{0.50,0.00,0.50}{##1}}}
\@namedef{PY@tok@gd}{\def\PY@tc##1{\textcolor[rgb]{0.63,0.00,0.00}{##1}}}
\@namedef{PY@tok@gi}{\def\PY@tc##1{\textcolor[rgb]{0.00,0.52,0.00}{##1}}}
\@namedef{PY@tok@gr}{\def\PY@tc##1{\textcolor[rgb]{0.89,0.00,0.00}{##1}}}
\@namedef{PY@tok@ge}{\let\PY@it=\textit}
\@namedef{PY@tok@gs}{\let\PY@bf=\textbf}
\@namedef{PY@tok@gp}{\let\PY@bf=\textbf\def\PY@tc##1{\textcolor[rgb]{0.00,0.00,0.50}{##1}}}
\@namedef{PY@tok@go}{\def\PY@tc##1{\textcolor[rgb]{0.44,0.44,0.44}{##1}}}
\@namedef{PY@tok@gt}{\def\PY@tc##1{\textcolor[rgb]{0.00,0.27,0.87}{##1}}}
\@namedef{PY@tok@err}{\def\PY@bc##1{{\setlength{\fboxsep}{\string -\fboxrule}\fcolorbox[rgb]{1.00,0.00,0.00}{1,1,1}{\strut ##1}}}}
\@namedef{PY@tok@kc}{\let\PY@bf=\textbf\def\PY@tc##1{\textcolor[rgb]{0.00,0.50,0.00}{##1}}}
\@namedef{PY@tok@kd}{\let\PY@bf=\textbf\def\PY@tc##1{\textcolor[rgb]{0.00,0.50,0.00}{##1}}}
\@namedef{PY@tok@kn}{\let\PY@bf=\textbf\def\PY@tc##1{\textcolor[rgb]{0.00,0.50,0.00}{##1}}}
\@namedef{PY@tok@kr}{\let\PY@bf=\textbf\def\PY@tc##1{\textcolor[rgb]{0.00,0.50,0.00}{##1}}}
\@namedef{PY@tok@bp}{\def\PY@tc##1{\textcolor[rgb]{0.00,0.50,0.00}{##1}}}
\@namedef{PY@tok@fm}{\def\PY@tc##1{\textcolor[rgb]{0.00,0.00,1.00}{##1}}}
\@namedef{PY@tok@vc}{\def\PY@tc##1{\textcolor[rgb]{0.10,0.09,0.49}{##1}}}
\@namedef{PY@tok@vg}{\def\PY@tc##1{\textcolor[rgb]{0.10,0.09,0.49}{##1}}}
\@namedef{PY@tok@vi}{\def\PY@tc##1{\textcolor[rgb]{0.10,0.09,0.49}{##1}}}
\@namedef{PY@tok@vm}{\def\PY@tc##1{\textcolor[rgb]{0.10,0.09,0.49}{##1}}}
\@namedef{PY@tok@sa}{\def\PY@tc##1{\textcolor[rgb]{0.73,0.13,0.13}{##1}}}
\@namedef{PY@tok@sb}{\def\PY@tc##1{\textcolor[rgb]{0.73,0.13,0.13}{##1}}}
\@namedef{PY@tok@sc}{\def\PY@tc##1{\textcolor[rgb]{0.73,0.13,0.13}{##1}}}
\@namedef{PY@tok@dl}{\def\PY@tc##1{\textcolor[rgb]{0.73,0.13,0.13}{##1}}}
\@namedef{PY@tok@s2}{\def\PY@tc##1{\textcolor[rgb]{0.73,0.13,0.13}{##1}}}
\@namedef{PY@tok@sh}{\def\PY@tc##1{\textcolor[rgb]{0.73,0.13,0.13}{##1}}}
\@namedef{PY@tok@s1}{\def\PY@tc##1{\textcolor[rgb]{0.73,0.13,0.13}{##1}}}
\@namedef{PY@tok@mb}{\def\PY@tc##1{\textcolor[rgb]{0.40,0.40,0.40}{##1}}}
\@namedef{PY@tok@mf}{\def\PY@tc##1{\textcolor[rgb]{0.40,0.40,0.40}{##1}}}
\@namedef{PY@tok@mh}{\def\PY@tc##1{\textcolor[rgb]{0.40,0.40,0.40}{##1}}}
\@namedef{PY@tok@mi}{\def\PY@tc##1{\textcolor[rgb]{0.40,0.40,0.40}{##1}}}
\@namedef{PY@tok@il}{\def\PY@tc##1{\textcolor[rgb]{0.40,0.40,0.40}{##1}}}
\@namedef{PY@tok@mo}{\def\PY@tc##1{\textcolor[rgb]{0.40,0.40,0.40}{##1}}}
\@namedef{PY@tok@ch}{\let\PY@it=\textit\def\PY@tc##1{\textcolor[rgb]{0.24,0.48,0.48}{##1}}}
\@namedef{PY@tok@cm}{\let\PY@it=\textit\def\PY@tc##1{\textcolor[rgb]{0.24,0.48,0.48}{##1}}}
\@namedef{PY@tok@cpf}{\let\PY@it=\textit\def\PY@tc##1{\textcolor[rgb]{0.24,0.48,0.48}{##1}}}
\@namedef{PY@tok@c1}{\let\PY@it=\textit\def\PY@tc##1{\textcolor[rgb]{0.24,0.48,0.48}{##1}}}
\@namedef{PY@tok@cs}{\let\PY@it=\textit\def\PY@tc##1{\textcolor[rgb]{0.24,0.48,0.48}{##1}}}

\def\PYZbs{\char`\\}
\def\PYZus{\char`\_}
\def\PYZob{\char`\{}
\def\PYZcb{\char`\}}
\def\PYZca{\char`\^}
\def\PYZam{\char`\&}
\def\PYZlt{\char`\<}
\def\PYZgt{\char`\>}
\def\PYZsh{\char`\#}
\def\PYZpc{\char`\%}
\def\PYZdl{\char`\$}
\def\PYZhy{\char`\-}
\def\PYZsq{\char`\'}
\def\PYZdq{\char`\"}
\def\PYZti{\char`\~}
% for compatibility with earlier versions
\def\PYZat{@}
\def\PYZlb{[}
\def\PYZrb{]}
\makeatother


    % For linebreaks inside Verbatim environment from package fancyvrb.
    \makeatletter
        \newbox\Wrappedcontinuationbox
        \newbox\Wrappedvisiblespacebox
        \newcommand*\Wrappedvisiblespace {\textcolor{red}{\textvisiblespace}}
        \newcommand*\Wrappedcontinuationsymbol {\textcolor{red}{\llap{\tiny$\m@th\hookrightarrow$}}}
        \newcommand*\Wrappedcontinuationindent {3ex }
        \newcommand*\Wrappedafterbreak {\kern\Wrappedcontinuationindent\copy\Wrappedcontinuationbox}
        % Take advantage of the already applied Pygments mark-up to insert
        % potential linebreaks for TeX processing.
        %        {, <, #, %, $, ' and ": go to next line.
        %        _, }, ^, &, >, - and ~: stay at end of broken line.
        % Use of \textquotesingle for straight quote.
        \newcommand*\Wrappedbreaksatspecials {%
            \def\PYGZus{\discretionary{\char`\_}{\Wrappedafterbreak}{\char`\_}}%
            \def\PYGZob{\discretionary{}{\Wrappedafterbreak\char`\{}{\char`\{}}%
            \def\PYGZcb{\discretionary{\char`\}}{\Wrappedafterbreak}{\char`\}}}%
            \def\PYGZca{\discretionary{\char`\^}{\Wrappedafterbreak}{\char`\^}}%
            \def\PYGZam{\discretionary{\char`\&}{\Wrappedafterbreak}{\char`\&}}%
            \def\PYGZlt{\discretionary{}{\Wrappedafterbreak\char`\<}{\char`\<}}%
            \def\PYGZgt{\discretionary{\char`\>}{\Wrappedafterbreak}{\char`\>}}%
            \def\PYGZsh{\discretionary{}{\Wrappedafterbreak\char`\#}{\char`\#}}%
            \def\PYGZpc{\discretionary{}{\Wrappedafterbreak\char`\%}{\char`\%}}%
            \def\PYGZdl{\discretionary{}{\Wrappedafterbreak\char`\$}{\char`\$}}%
            \def\PYGZhy{\discretionary{\char`\-}{\Wrappedafterbreak}{\char`\-}}%
            \def\PYGZsq{\discretionary{}{\Wrappedafterbreak\textquotesingle}{\textquotesingle}}%
            \def\PYGZdq{\discretionary{}{\Wrappedafterbreak\char`\"}{\char`\"}}%
            \def\PYGZti{\discretionary{\char`\~}{\Wrappedafterbreak}{\char`\~}}%
        }
        % Some characters . , ; ? ! / are not pygmentized.
        % This macro makes them "active" and they will insert potential linebreaks
        \newcommand*\Wrappedbreaksatpunct {%
            \lccode`\~`\.\lowercase{\def~}{\discretionary{\hbox{\char`\.}}{\Wrappedafterbreak}{\hbox{\char`\.}}}%
            \lccode`\~`\,\lowercase{\def~}{\discretionary{\hbox{\char`\,}}{\Wrappedafterbreak}{\hbox{\char`\,}}}%
            \lccode`\~`\;\lowercase{\def~}{\discretionary{\hbox{\char`\;}}{\Wrappedafterbreak}{\hbox{\char`\;}}}%
            \lccode`\~`\:\lowercase{\def~}{\discretionary{\hbox{\char`\:}}{\Wrappedafterbreak}{\hbox{\char`\:}}}%
            \lccode`\~`\?\lowercase{\def~}{\discretionary{\hbox{\char`\?}}{\Wrappedafterbreak}{\hbox{\char`\?}}}%
            \lccode`\~`\!\lowercase{\def~}{\discretionary{\hbox{\char`\!}}{\Wrappedafterbreak}{\hbox{\char`\!}}}%
            \lccode`\~`\/\lowercase{\def~}{\discretionary{\hbox{\char`\/}}{\Wrappedafterbreak}{\hbox{\char`\/}}}%
            \catcode`\.\active
            \catcode`\,\active
            \catcode`\;\active
            \catcode`\:\active
            \catcode`\?\active
            \catcode`\!\active
            \catcode`\/\active
            \lccode`\~`\~
        }
    \makeatother

    \let\OriginalVerbatim=\Verbatim
    \makeatletter
    \renewcommand{\Verbatim}[1][1]{%
        %\parskip\z@skip
        \sbox\Wrappedcontinuationbox {\Wrappedcontinuationsymbol}%
        \sbox\Wrappedvisiblespacebox {\FV@SetupFont\Wrappedvisiblespace}%
        \def\FancyVerbFormatLine ##1{\hsize\linewidth
            \vtop{\raggedright\hyphenpenalty\z@\exhyphenpenalty\z@
                \doublehyphendemerits\z@\finalhyphendemerits\z@
                \strut ##1\strut}%
        }%
        % If the linebreak is at a space, the latter will be displayed as visible
        % space at end of first line, and a continuation symbol starts next line.
        % Stretch/shrink are however usually zero for typewriter font.
        \def\FV@Space {%
            \nobreak\hskip\z@ plus\fontdimen3\font minus\fontdimen4\font
            \discretionary{\copy\Wrappedvisiblespacebox}{\Wrappedafterbreak}
            {\kern\fontdimen2\font}%
        }%

        % Allow breaks at special characters using \PYG... macros.
        \Wrappedbreaksatspecials
        % Breaks at punctuation characters . , ; ? ! and / need catcode=\active
        \OriginalVerbatim[#1,codes*=\Wrappedbreaksatpunct]%
    }
    \makeatother

    % Exact colors from NB
    \definecolor{incolor}{HTML}{303F9F}
    \definecolor{outcolor}{HTML}{D84315}
    \definecolor{cellborder}{HTML}{CFCFCF}
    \definecolor{cellbackground}{HTML}{F7F7F7}

    % prompt
    \makeatletter
    \newcommand{\boxspacing}{\kern\kvtcb@left@rule\kern\kvtcb@boxsep}
    \makeatother
    \newcommand{\prompt}[4]{
        {\ttfamily\llap{{\color{#2}[#3]:\hspace{3pt}#4}}\vspace{-\baselineskip}}
    }
    

    
    % Prevent overflowing lines due to hard-to-break entities
    \sloppy
    % Setup hyperref package
    \hypersetup{
      breaklinks=true,  % so long urls are correctly broken across lines
      colorlinks=true,
      urlcolor=urlcolor,
      linkcolor=linkcolor,
      citecolor=citecolor,
      }
    % Slightly bigger margins than the latex defaults
    
    \geometry{verbose,tmargin=1in,bmargin=1in,lmargin=1in,rmargin=1in}
    
    

\begin{document}
    
    \maketitle
    
    

    
    \hypertarget{project-1---object-detection-in-urban-environment}{%
\section{\texorpdfstring{\textbf{PROJECT 1 - OBJECT DETECTION IN URBAN
ENVIRONMENT}}{PROJECT 1 - OBJECT DETECTION IN URBAN ENVIRONMENT}}\label{project-1---object-detection-in-urban-environment}}

    \hypertarget{roadmap}{%
\subsection{\texorpdfstring{\textbf{ROADMAP}}{ROADMAP}}\label{roadmap}}

\begin{itemize}
\item
  Install the TensorFlow Object Detection API.
\item
  Edit the model pipeline config file and download the pre-trained model
  checkpoint.
\item
  Train and evaluate the model.
\item
  Output a video with detections
\end{itemize}

    \hypertarget{import-libraries}{%
\section{\texorpdfstring{\textbf{1) Import
Libraries}}{1) Import Libraries}}\label{import-libraries}}

    \begin{tcolorbox}[breakable, size=fbox, boxrule=1pt, pad at break*=1mm,colback=cellbackground, colframe=cellborder]
\prompt{In}{incolor}{ }{\boxspacing}
\begin{Verbatim}[commandchars=\\\{\}]
\PY{k+kn}{import} \PY{n+nn}{os}
\PY{k+kn}{import} \PY{n+nn}{glob}
\PY{k+kn}{import} \PY{n+nn}{pathlib}
\PY{k+kn}{import} \PY{n+nn}{pprint}
\PY{k+kn}{import} \PY{n+nn}{xml}\PY{n+nn}{.}\PY{n+nn}{etree}\PY{n+nn}{.}\PY{n+nn}{ElementTree} \PY{k}{as} \PY{n+nn}{ET}
\PY{k+kn}{import} \PY{n+nn}{pandas} \PY{k}{as} \PY{n+nn}{pd}
\PY{k+kn}{import} \PY{n+nn}{tensorflow} \PY{k}{as} \PY{n+nn}{tf}
\PY{n+nb}{print}\PY{p}{(}\PY{n}{tf}\PY{o}{.}\PY{n}{\PYZus{}\PYZus{}version\PYZus{}\PYZus{}}\PY{p}{)}
\end{Verbatim}
\end{tcolorbox}

    \hypertarget{mount-drive-and-link-your-folders}{%
\section{\texorpdfstring{\textbf{2) Mount drive and link your
folders}}{2) Mount drive and link your folders}}\label{mount-drive-and-link-your-folders}}

    \begin{tcolorbox}[breakable, size=fbox, boxrule=1pt, pad at break*=1mm,colback=cellbackground, colframe=cellborder]
\prompt{In}{incolor}{ }{\boxspacing}
\begin{Verbatim}[commandchars=\\\{\}]
\PY{k+kn}{from} \PY{n+nn}{google}\PY{n+nn}{.}\PY{n+nn}{colab} \PY{k+kn}{import} \PY{n}{drive}
\PY{n}{drive}\PY{o}{.}\PY{n}{mount}\PY{p}{(}\PY{l+s+s1}{\PYZsq{}}\PY{l+s+s1}{/content/gdrive}\PY{l+s+s1}{\PYZsq{}}\PY{p}{)}

\PY{c+c1}{\PYZsh{} this creates a symbolic link so that now the path /content/gdrive/My\PYZbs{} Drive/ is equal to /mydrive}
\PY{o}{!}ln\PY{+w}{ }\PYZhy{}s\PY{+w}{ }/content/gdrive/My\PY{l+s+se}{\PYZbs{} }Drive/\PY{+w}{ }/mydrive
\PY{o}{!}ls\PY{+w}{ }/mydrive
\end{Verbatim}
\end{tcolorbox}

    \hypertarget{clone-the-tensorflow-models-git-repository-install-tensorflow-object-detection-api}{%
\section{\texorpdfstring{\textbf{3) Clone the tensorflow models git
repository \& Install TensorFlow Object Detection
API}}{3) Clone the tensorflow models git repository \& Install TensorFlow Object Detection API}}\label{clone-the-tensorflow-models-git-repository-install-tensorflow-object-detection-api}}

    \begin{tcolorbox}[breakable, size=fbox, boxrule=1pt, pad at break*=1mm,colback=cellbackground, colframe=cellborder]
\prompt{In}{incolor}{ }{\boxspacing}
\begin{Verbatim}[commandchars=\\\{\}]
\PY{c+c1}{\PYZsh{} clone the tensorflow models on the colab cloud vm}
\PY{o}{!}git\PY{+w}{ }clone\PY{+w}{ }\PYZhy{}\PYZhy{}q\PY{+w}{ }https://github.com/tensorflow/models.git

\PY{c+c1}{\PYZsh{}navigate to /models/research folder to compile protos}
\PY{o}{\PYZpc{}}\PY{k}{cd} models/research

\PY{c+c1}{\PYZsh{} Compile protos.}
\PY{o}{!}protoc\PY{+w}{ }object\PYZus{}detection/protos/*.proto\PY{+w}{ }\PYZhy{}\PYZhy{}python\PYZus{}out\PY{o}{=}.

\PY{c+c1}{\PYZsh{} Install TensorFlow Object Detection API.}
\PY{o}{!}cp\PY{+w}{ }object\PYZus{}detection/packages/tf2/setup.py\PY{+w}{ }.
\PY{o}{!}python\PY{+w}{ }\PYZhy{}m\PY{+w}{ }pip\PY{+w}{ }install\PY{+w}{ }.
\end{Verbatim}
\end{tcolorbox}

    \hypertarget{test-the-model-builder}{%
\section{\texorpdfstring{\textbf{4) Test the model
builder}}{4) Test the model builder}}\label{test-the-model-builder}}

    \begin{tcolorbox}[breakable, size=fbox, boxrule=1pt, pad at break*=1mm,colback=cellbackground, colframe=cellborder]
\prompt{In}{incolor}{ }{\boxspacing}
\begin{Verbatim}[commandchars=\\\{\}]
\PY{c+c1}{\PYZsh{} Testing the model builder}
\PY{o}{!}python\PY{+w}{ }object\PYZus{}detection/builders/model\PYZus{}builder\PYZus{}tf2\PYZus{}test.py
\end{Verbatim}
\end{tcolorbox}

    \hypertarget{download-pre-trained-model-checkpoint}{%
\section{\texorpdfstring{\textbf{5) Download pre-trained model
checkpoint}}{5) Download pre-trained model checkpoint}}\label{download-pre-trained-model-checkpoint}}

Download \textbf{the model} into the \textbf{data} folder \& unzip it.

A list of detection checkpoints for tensorflow 2.x can be found
\href{https://github.com/tensorflow/models/blob/master/research/object_detection/g3doc/tf2_detection_zoo.md}{here}.

    \begin{tcolorbox}[breakable, size=fbox, boxrule=1pt, pad at break*=1mm,colback=cellbackground, colframe=cellborder]
\prompt{In}{incolor}{ }{\boxspacing}
\begin{Verbatim}[commandchars=\\\{\}]
\PY{c+c1}{\PYZsh{} Working directory}

\PY{o}{\PYZpc{}}\PY{k}{cd} /mydrive/Project/customTF2/data/
\end{Verbatim}
\end{tcolorbox}

    \begin{tcolorbox}[breakable, size=fbox, boxrule=1pt, pad at break*=1mm,colback=cellbackground, colframe=cellborder]
\prompt{In}{incolor}{ }{\boxspacing}
\begin{Verbatim}[commandchars=\\\{\}]
\PY{c+c1}{\PYZsh{}Download the pre\PYZhy{}trained model into the data folder \PYZam{} unzip it.}

\PY{c+c1}{\PYZsh{} !wget \PYZhy{}\PYZlt{} Link of the pre\PYZhy{}trained model \PYZgt{}\PYZhy{}}
\PY{o}{!}wget\PY{+w}{ }http://download.tensorflow.org/models/object\PYZus{}detection/tf2/20200711/efficientdet\PYZus{}d1\PYZus{}coco17\PYZus{}tpu\PYZhy{}32.tar.gz

\PY{c+c1}{\PYZsh{} !tar \PYZhy{}xzvf \PYZhy{}\PYZlt{}Name\PYZus{}of\PYZus{}file.tar.gz\PYZgt{}\PYZhy{}}
\PY{o}{!}tar\PY{+w}{ }\PYZhy{}xzvf\PY{+w}{ }efficientdet\PYZus{}d1\PYZus{}coco17\PYZus{}tpu\PYZhy{}32.tar.gz
\end{Verbatim}
\end{tcolorbox}

    \hypertarget{make-changes-to-the-model-pipeline-config-file}{%
\section{\texorpdfstring{\textbf{6) Make changes to the model pipeline
config
file}}{6) Make changes to the model pipeline config file}}\label{make-changes-to-the-model-pipeline-config-file}}

Current working directory is /mydrive/Project/customTF2/data/(Model
File)

\textbf{You need to make the following changes:} * change
\textbf{\emph{num\_classes}} to number of your classes. * change
\textbf{\emph{test.record}} path, \textbf{\emph{train.record}} path \&
\textbf{\emph{labelmap}} path to the paths where you have created these
files (paths should be relative to your current working directory while
training). * change \textbf{\emph{fine\_tune\_checkpoint}} to the path
of the directory where the downloaded checkpoint. * change
\textbf{\emph{fine\_tune\_checkpoint\_type}} with value
\textbf{classification} or \textbf{detection} depending on the type.. *
change \textbf{\emph{batch\_size}} to any multiple of 8 depending upon
the capability of your GPU. (eg:- 24,128,\ldots,512). * change
\textbf{\emph{num\_steps}} to number of steps you want the detector to
train.

    \hypertarget{load-tensorboard}{%
\section{\texorpdfstring{\textbf{7) Load
Tensorboard}}{7) Load Tensorboard}}\label{load-tensorboard}}

    \begin{tcolorbox}[breakable, size=fbox, boxrule=1pt, pad at break*=1mm,colback=cellbackground, colframe=cellborder]
\prompt{In}{incolor}{ }{\boxspacing}
\begin{Verbatim}[commandchars=\\\{\}]
\PY{c+c1}{\PYZsh{} Load TensorBoard}

\PY{o}{\PYZpc{}}\PY{k}{load\PYZus{}ext} tensorboard
\PY{o}{\PYZpc{}}\PY{k}{tensorboard} \PYZhy{}\PYZhy{}logdir \PYZsq{}/content/gdrive/MyDrive/Project/customTF2/logs/resnet\PYZus{}logs\PYZsq{}
\end{Verbatim}
\end{tcolorbox}

    \begin{tcolorbox}[breakable, size=fbox, boxrule=1pt, pad at break*=1mm,colback=cellbackground, colframe=cellborder]
\prompt{In}{incolor}{ }{\boxspacing}
\begin{Verbatim}[commandchars=\\\{\}]
\PY{c+c1}{\PYZsh{}\PYZsh{} Incase an error occurs with TensorBoard ***UNCOMMENT*** the following lines and run it to find this session PID and terminate it using !kill}

\PY{c+c1}{\PYZsh{} from tensorboard import notebook}
\PY{c+c1}{\PYZsh{} notebook.list() \PYZsh{} View open TensorBoard instances}
\end{Verbatim}
\end{tcolorbox}

    \begin{tcolorbox}[breakable, size=fbox, boxrule=1pt, pad at break*=1mm,colback=cellbackground, colframe=cellborder]
\prompt{In}{incolor}{ }{\boxspacing}
\begin{Verbatim}[commandchars=\\\{\}]
\PY{c+c1}{\PYZsh{} !kill 39390}
\end{Verbatim}
\end{tcolorbox}

    \hypertarget{train-the-model}{%
\section{\texorpdfstring{\textbf{8) Train the
model}}{8) Train the model}}\label{train-the-model}}

    \hypertarget{navigate-to-the-object_detection-folder-in-colab-vm}{%
\subsection{\texorpdfstring{Navigate to the
\textbf{\emph{object\_detection}} folder in colab
vm}{Navigate to the object\_detection folder in colab vm}}\label{navigate-to-the-object_detection-folder-in-colab-vm}}

    \begin{tcolorbox}[breakable, size=fbox, boxrule=1pt, pad at break*=1mm,colback=cellbackground, colframe=cellborder]
\prompt{In}{incolor}{ }{\boxspacing}
\begin{Verbatim}[commandchars=\\\{\}]
\PY{o}{\PYZpc{}}\PY{k}{cd} /content/models/research/object\PYZus{}detection
\end{Verbatim}
\end{tcolorbox}

    \hypertarget{i---training-using-model_main_tf2.py}{%
\subsection{I - Training using
model\_main\_tf2.py}\label{i---training-using-model_main_tf2.py}}

Here \textbf{\{PIPELINE\_CONFIG\_PATH\}} points to the pipeline config
and \textbf{\{MODEL\_DIR\}} points to the directory in which training
checkpoints and events will be written.

For best results, you should stop the training when the loss is less
than 0.1 if possible, else train the model until the loss does not show
any significant change for a while. The ideal loss should be below 0.05
(Try to get the loss as low as possible without overfitting the model.
Don't go too high on training steps to try and lower the loss if the
model has already converged viz.~if it does not reduce loss
significantly any further and takes a while to go down. )

    \begin{tcolorbox}[breakable, size=fbox, boxrule=1pt, pad at break*=1mm,colback=cellbackground, colframe=cellborder]
\prompt{In}{incolor}{ }{\boxspacing}
\begin{Verbatim}[commandchars=\\\{\}]
\PY{c+c1}{\PYZsh{} Run the command below from the content/models/research/object\PYZus{}detection directory}
\PY{l+s+sd}{\PYZdq{}\PYZdq{}\PYZdq{}}
\PY{l+s+sd}{PIPELINE\PYZus{}CONFIG\PYZus{}PATH=path/to/pipeline.config}
\PY{l+s+sd}{MODEL\PYZus{}DIR=path to training checkpoints directory}
\PY{l+s+sd}{NUM\PYZus{}TRAIN\PYZus{}STEPS=2000}
\PY{l+s+sd}{SAMPLE\PYZus{}1\PYZus{}OF\PYZus{}N\PYZus{}EVAL\PYZus{}EXAMPLES=1}

\PY{l+s+sd}{python model\PYZus{}main\PYZus{}tf2.py \PYZhy{}\PYZhy{} \PYZbs{}}
\PY{l+s+sd}{  \PYZhy{}\PYZhy{}model\PYZus{}dir=\PYZdl{}MODEL\PYZus{}DIR \PYZhy{}\PYZhy{}num\PYZus{}train\PYZus{}steps=\PYZdl{}NUM\PYZus{}TRAIN\PYZus{}STEPS \PYZbs{}}
\PY{l+s+sd}{  \PYZhy{}\PYZhy{}sample\PYZus{}1\PYZus{}of\PYZus{}n\PYZus{}eval\PYZus{}examples=\PYZdl{}SAMPLE\PYZus{}1\PYZus{}OF\PYZus{}N\PYZus{}EVAL\PYZus{}EXAMPLES \PYZbs{}}
\PY{l+s+sd}{  \PYZhy{}\PYZhy{}pipeline\PYZus{}config\PYZus{}path=\PYZdl{}PIPELINE\PYZus{}CONFIG\PYZus{}PATH \PYZbs{}}
\PY{l+s+sd}{  \PYZhy{}\PYZhy{}alsologtostderr}
\PY{l+s+sd}{\PYZdq{}\PYZdq{}\PYZdq{}}

\PY{o}{!}python\PY{+w}{ }model\PYZus{}main\PYZus{}tf2.py\PY{+w}{ }\PYZhy{}\PYZhy{}pipeline\PYZus{}config\PYZus{}path\PY{o}{=}/content/gdrive/MyDrive/Project/customTF2/data/ssd\PYZus{}resnet50\PYZus{}v1\PYZus{}fpn\PYZus{}640x640\PYZus{}coco17\PYZus{}tpu\PYZhy{}8/pipeline.config\PY{+w}{ }\PYZhy{}\PYZhy{}model\PYZus{}dir\PY{o}{=}/content/gdrive/MyDrive/Project/customTF2/logs/resnet\PYZus{}logs\PY{+w}{ }\PYZhy{}\PYZhy{}alsologtostderr
\end{Verbatim}
\end{tcolorbox}

    \hypertarget{troubleshooting}{%
\subsubsection{\texorpdfstring{\textbf{TROUBLESHOOTING:}}{TROUBLESHOOTING:}}\label{troubleshooting}}

If you get an error for \_registerMatType cv2 above, this might be
because of OpenCV version mismatches in Colab. Run
\texttt{!pip\ list\textbar{}grep\ opencv} to see the versions of OpenCV
packages installed i.e.~\texttt{opencv-python},
\texttt{opencv-contrib-python} \& \texttt{opencv-python-headless}. The
versions will be different which is causing this error. This error will
go away when colab updates it supported versions. For now, you can fix
this by simply uninstalling and installing OpenCV packages.

\textbf{Check versions:}

!pip list\textbar grep opencv

\textbf{Use the following 2 commands if only the opencv-python-headless
is of different version}:

!pip uninstall opencv-python-headless --y

!pip install opencv-python-headless==4.1.2.30

\textbf{Or use the following commands if other opencv packages are of
different versions. Uninstall and install all with the same version}:

!pip uninstall opencv-python --y

!pip uninstall opencv-contrib-python --y

!pip uninstall opencv-python-headless --y

!pip install opencv-python==4.5.4.60

!pip install opencv-contrib-python==4.5.4.60

!pip install opencv-python-headless==4.5.4.60

    \hypertarget{ii---evaluation-using-model_main_tf2.py}{%
\subsection{II - Evaluation using
model\_main\_tf2.py}\label{ii---evaluation-using-model_main_tf2.py}}

You can run this in parallel by opening another colab notebook and
running this command simultaneously along with the training command
above (don't forget to mount drive, clone the TF git repo and install
the TF2 object detection API there as well). This will give you
validation loss, mAP, etc so you have a better idea of how your model is
performing.

Here \textbf{\{CHECKPOINT\_DIR\}} points to the directory with
checkpoints produced by the training job. Evaluation events are written
to \textbf{\{MODEL\_DIR/eval\}}.

    \begin{tcolorbox}[breakable, size=fbox, boxrule=1pt, pad at break*=1mm,colback=cellbackground, colframe=cellborder]
\prompt{In}{incolor}{ }{\boxspacing}
\begin{Verbatim}[commandchars=\\\{\}]
\PY{c+c1}{\PYZsh{} Run the command below from the content/models/research/object\PYZus{}detection directory}
\PY{l+s+sd}{\PYZdq{}\PYZdq{}\PYZdq{}}
\PY{l+s+sd}{PIPELINE\PYZus{}CONFIG\PYZus{}PATH=path/to/pipeline.config}
\PY{l+s+sd}{MODEL\PYZus{}DIR=path to training checkpoints directory}
\PY{l+s+sd}{CHECKPOINT\PYZus{}DIR=\PYZdl{}\PYZob{}MODEL\PYZus{}DIR\PYZcb{}}
\PY{l+s+sd}{NUM\PYZus{}TRAIN\PYZus{}STEPS=2000}
\PY{l+s+sd}{SAMPLE\PYZus{}1\PYZus{}OF\PYZus{}N\PYZus{}EVAL\PYZus{}EXAMPLES=1}

\PY{l+s+sd}{python model\PYZus{}main\PYZus{}tf2.py \PYZhy{}\PYZhy{} \PYZbs{}}
\PY{l+s+sd}{  \PYZhy{}\PYZhy{}model\PYZus{}dir=\PYZdl{}MODEL\PYZus{}DIR \PYZhy{}\PYZhy{}num\PYZus{}train\PYZus{}steps=\PYZdl{}NUM\PYZus{}TRAIN\PYZus{}STEPS \PYZbs{}}
\PY{l+s+sd}{  \PYZhy{}\PYZhy{}checkpoint\PYZus{}dir=\PYZdl{}\PYZob{}CHECKPOINT\PYZus{}DIR\PYZcb{} \PYZbs{}}
\PY{l+s+sd}{  \PYZhy{}\PYZhy{}sample\PYZus{}1\PYZus{}of\PYZus{}n\PYZus{}eval\PYZus{}examples=\PYZdl{}SAMPLE\PYZus{}1\PYZus{}OF\PYZus{}N\PYZus{}EVAL\PYZus{}EXAMPLES \PYZbs{}}
\PY{l+s+sd}{  \PYZhy{}\PYZhy{}pipeline\PYZus{}config\PYZus{}path=\PYZdl{}PIPELINE\PYZus{}CONFIG\PYZus{}PATH \PYZbs{}}
\PY{l+s+sd}{  \PYZhy{}\PYZhy{}alsologtostderr}
\PY{l+s+sd}{\PYZdq{}\PYZdq{}\PYZdq{}}

\PY{o}{!}python\PY{+w}{ }model\PYZus{}main\PYZus{}tf2.py\PY{+w}{ }\PYZhy{}\PYZhy{}pipeline\PYZus{}config\PYZus{}path\PY{o}{=}/mydrive/Project/customTF2/data/ssd\PYZus{}resnet50\PYZus{}v1\PYZus{}fpn\PYZus{}640x640\PYZus{}coco17\PYZus{}tpu\PYZhy{}8/pipeline.config\PY{+w}{ }\PYZhy{}\PYZhy{}model\PYZus{}dir\PY{o}{=}/mydrive/Project/customTF2/logs/resnet\PYZus{}logs\PY{+w}{ }\PYZhy{}\PYZhy{}alsologtostderr
\end{Verbatim}
\end{tcolorbox}

    \hypertarget{retraining-the-model-in-case-you-get-disconnected}{%
\subsection{RETRAINING THE MODEL ( in case you get disconnected
)}\label{retraining-the-model-in-case-you-get-disconnected}}

If you get disconnected or lose your session on colab vm, you can start
your training where you left off as the checkpoint is saved on your
drive inside the \textbf{\emph{training}} folder. To restart the
training simply run \textbf{steps 1 till 7.}

Note that since we have all the files required for training like the
record files,our edited pipeline config file,the label\_map file and the
model checkpoint folder, therefore we do not need to create these again.

\textbf{The model\_main\_tf2.py script saves the checkpoint every 1000
steps.} The training automatically restarts from the last saved
checkpoint itself.

However, if you see that it doesn't restart training from the last
checkpoint you can make 1 change in the pipeline config file. Change
\textbf{fine\_tune\_checkpoint} to where your latest trained checkpoints
have been written and have it point to the latest checkpoint as shown
below:

\begin{verbatim}
fine_tune_checkpoint: "/mydrive/Project/customTF2/logs/ckpt-X" (where ckpt-X is the latest checkpoint)
\end{verbatim}

    \hypertarget{test-your-trained-model}{%
\section{\texorpdfstring{\textbf{9) Test your trained
model}}{9) Test your trained model}}\label{test-your-trained-model}}

    \hypertarget{export-inference-graph}{%
\subsection{Export inference graph}\label{export-inference-graph}}

Current working directory is /content/models/research/object\_detection

\textbf{\emph{CHANGE}} \textless{} Model\_File \textgreater{} in
\textless{} --pipeline\_config\_path \textgreater{} to the name of you
model folder

    \begin{tcolorbox}[breakable, size=fbox, boxrule=1pt, pad at break*=1mm,colback=cellbackground, colframe=cellborder]
\prompt{In}{incolor}{ }{\boxspacing}
\begin{Verbatim}[commandchars=\\\{\}]
\PY{c+c1}{\PYZsh{} Export inference graph}

\PY{o}{!}python\PY{+w}{ }exporter\PYZus{}main\PYZus{}v2.py\PY{+w}{ }\PYZhy{}\PYZhy{}trained\PYZus{}checkpoint\PYZus{}dir\PY{o}{=}/content/gdrive/MyDrive/Project/customTF2/logs\PY{+w}{ }\PYZhy{}\PYZhy{}pipeline\PYZus{}config\PYZus{}path\PY{o}{=}/content/gdrive/MyDrive/Project/customTF2/data/\PYZlt{}\PY{+w}{ }Model\PYZus{}File\PY{+w}{ }\PYZgt{}/pipeline.config\PY{+w}{ }\PYZhy{}\PYZhy{}output\PYZus{}directory\PY{+w}{ }/mydrive/Project/customTF2/data/inference\PYZus{}graph\PYZus{}resnet
\end{Verbatim}
\end{tcolorbox}

    \hypertarget{test-your-trained-object-detection-model-on-a-video}{%
\subsection{Test your trained Object Detection model on a
video}\label{test-your-trained-object-detection-model-on-a-video}}

    \begin{tcolorbox}[breakable, size=fbox, boxrule=1pt, pad at break*=1mm,colback=cellbackground, colframe=cellborder]
\prompt{In}{incolor}{ }{\boxspacing}
\begin{Verbatim}[commandchars=\\\{\}]
\PY{c+c1}{\PYZsh{} Importing libraries}

\PY{k+kn}{import} \PY{n+nn}{tensorflow} \PY{k}{as} \PY{n+nn}{tf}
\PY{k+kn}{import} \PY{n+nn}{time}
\PY{k+kn}{import} \PY{n+nn}{numpy} \PY{k}{as} \PY{n+nn}{np}
\PY{k+kn}{import} \PY{n+nn}{warnings}
\PY{n}{warnings}\PY{o}{.}\PY{n}{filterwarnings}\PY{p}{(}\PY{l+s+s1}{\PYZsq{}}\PY{l+s+s1}{ignore}\PY{l+s+s1}{\PYZsq{}}\PY{p}{)}
\PY{k+kn}{from} \PY{n+nn}{PIL} \PY{k+kn}{import} \PY{n}{Image}
\PY{k+kn}{from} \PY{n+nn}{google}\PY{n+nn}{.}\PY{n+nn}{colab}\PY{n+nn}{.}\PY{n+nn}{patches} \PY{k+kn}{import} \PY{n}{cv2\PYZus{}imshow}
\PY{k+kn}{from} \PY{n+nn}{object\PYZus{}detection}\PY{n+nn}{.}\PY{n+nn}{utils} \PY{k+kn}{import} \PY{n}{label\PYZus{}map\PYZus{}util}
\PY{k+kn}{from} \PY{n+nn}{object\PYZus{}detection}\PY{n+nn}{.}\PY{n+nn}{utils} \PY{k+kn}{import} \PY{n}{visualization\PYZus{}utils} \PY{k}{as} \PY{n}{viz\PYZus{}utils}
\PY{k+kn}{import} \PY{n+nn}{cv2}
\end{Verbatim}
\end{tcolorbox}

    \begin{tcolorbox}[breakable, size=fbox, boxrule=1pt, pad at break*=1mm,colback=cellbackground, colframe=cellborder]
\prompt{In}{incolor}{ }{\boxspacing}
\begin{Verbatim}[commandchars=\\\{\}]
\PY{c+c1}{\PYZsh{} Output display size as you want}

\PY{n}{IMAGE\PYZus{}SIZE} \PY{o}{=} \PY{p}{(}\PY{l+m+mi}{12}\PY{p}{,} \PY{l+m+mi}{8}\PY{p}{)}
\end{Verbatim}
\end{tcolorbox}

    \begin{tcolorbox}[breakable, size=fbox, boxrule=1pt, pad at break*=1mm,colback=cellbackground, colframe=cellborder]
\prompt{In}{incolor}{ }{\boxspacing}
\begin{Verbatim}[commandchars=\\\{\}]
\PY{c+c1}{\PYZsh{} Load the model}

\PY{n}{PATH\PYZus{}TO\PYZus{}SAVED\PYZus{}MODEL}\PY{o}{=}\PY{l+s+s2}{\PYZdq{}}\PY{l+s+s2}{/content/gdrive/MyDrive/Project/customTF2/data/inference\PYZus{}graph/saved\PYZus{}model}\PY{l+s+s2}{\PYZdq{}}
\PY{n}{detect\PYZus{}fn}\PY{o}{=}\PY{n}{tf}\PY{o}{.}\PY{n}{saved\PYZus{}model}\PY{o}{.}\PY{n}{load}\PY{p}{(}\PY{n}{PATH\PYZus{}TO\PYZus{}SAVED\PYZus{}MODEL}\PY{p}{)}
\PY{n+nb}{print}\PY{p}{(}\PY{l+s+s1}{\PYZsq{}}\PY{l+s+s1}{Done!}\PY{l+s+s1}{\PYZsq{}}\PY{p}{)}
\end{Verbatim}
\end{tcolorbox}

    \begin{tcolorbox}[breakable, size=fbox, boxrule=1pt, pad at break*=1mm,colback=cellbackground, colframe=cellborder]
\prompt{In}{incolor}{ }{\boxspacing}
\begin{Verbatim}[commandchars=\\\{\}]
\PY{n}{category\PYZus{}index} \PY{o}{=} \PY{p}{\PYZob{}}
                    \PY{l+m+mi}{1}\PY{p}{:}\PY{p}{\PYZob{}}\PY{l+s+s1}{\PYZsq{}}\PY{l+s+s1}{id}\PY{l+s+s1}{\PYZsq{}}\PY{p}{:} \PY{l+m+mi}{1}\PY{p}{,} \PY{l+s+s1}{\PYZsq{}}\PY{l+s+s1}{name}\PY{l+s+s1}{\PYZsq{}}\PY{p}{:} \PY{l+s+s1}{\PYZsq{}}\PY{l+s+s1}{vehicle}\PY{l+s+s1}{\PYZsq{}}\PY{p}{\PYZcb{}}\PY{p}{,} 
                    \PY{l+m+mi}{2}\PY{p}{:} \PY{p}{\PYZob{}}\PY{l+s+s1}{\PYZsq{}}\PY{l+s+s1}{id}\PY{l+s+s1}{\PYZsq{}}\PY{p}{:} \PY{l+m+mi}{2}\PY{p}{,} \PY{l+s+s1}{\PYZsq{}}\PY{l+s+s1}{name}\PY{l+s+s1}{\PYZsq{}}\PY{p}{:} \PY{l+s+s1}{\PYZsq{}}\PY{l+s+s1}{pedestrian}\PY{l+s+s1}{\PYZsq{}}\PY{p}{\PYZcb{}}\PY{p}{,}
                    \PY{l+m+mi}{4}\PY{p}{:} \PY{p}{\PYZob{}}\PY{l+s+s1}{\PYZsq{}}\PY{l+s+s1}{id}\PY{l+s+s1}{\PYZsq{}}\PY{p}{:} \PY{l+m+mi}{4}\PY{p}{,} \PY{l+s+s1}{\PYZsq{}}\PY{l+s+s1}{name}\PY{l+s+s1}{\PYZsq{}}\PY{p}{:} \PY{l+s+s1}{\PYZsq{}}\PY{l+s+s1}{cyclist}\PY{l+s+s1}{\PYZsq{}}\PY{p}{\PYZcb{}}
                \PY{p}{\PYZcb{}}
\end{Verbatim}
\end{tcolorbox}

    \begin{tcolorbox}[breakable, size=fbox, boxrule=1pt, pad at break*=1mm,colback=cellbackground, colframe=cellborder]
\prompt{In}{incolor}{ }{\boxspacing}
\begin{Verbatim}[commandchars=\\\{\}]
\PY{c+c1}{\PYZsh{} Convert each frame into a np array and save it in list \PYZsq{}x\PYZsq{}}

\PY{n}{frames\PYZus{}path} \PY{o}{=} \PY{n+nb}{sorted}\PY{p}{(}\PY{n}{glob}\PY{o}{.}\PY{n}{glob}\PY{p}{(}\PY{l+s+s1}{\PYZsq{}}\PY{l+s+s1}{/mydrive/Project/test\PYZus{}video/*.png}\PY{l+s+s1}{\PYZsq{}}\PY{p}{)}\PY{p}{,} 
                     \PY{n}{key} \PY{o}{=} \PY{k}{lambda} \PY{n}{k}\PY{p}{:} \PY{n+nb}{int}\PY{p}{(}\PY{n}{os}\PY{o}{.}\PY{n}{path}\PY{o}{.}\PY{n}{basename}\PY{p}{(}\PY{n}{k}\PY{p}{)}\PY{o}{.}\PY{n}{split}\PY{p}{(}\PY{l+s+s1}{\PYZsq{}}\PY{l+s+s1}{.}\PY{l+s+s1}{\PYZsq{}}\PY{p}{)}\PY{p}{[}\PY{l+m+mi}{0}\PY{p}{]}\PY{o}{.}\PY{n}{split}\PY{p}{(}\PY{l+s+s1}{\PYZsq{}}\PY{l+s+s1}{\PYZus{}}\PY{l+s+s1}{\PYZsq{}}\PY{p}{)}\PY{p}{[}\PY{l+m+mi}{1}\PY{p}{]}\PY{p}{)}\PY{p}{)}

\PY{n}{x} \PY{o}{=} \PY{n}{np}\PY{o}{.}\PY{n}{array}\PY{p}{(}\PY{p}{[}\PY{n}{np}\PY{o}{.}\PY{n}{array}\PY{p}{(}\PY{n}{Image}\PY{o}{.}\PY{n}{open}\PY{p}{(}\PY{n}{fname}\PY{p}{)}\PY{p}{)} \PY{k}{for} \PY{n}{fname} \PY{o+ow}{in} \PY{n}{frames\PYZus{}path}\PY{p}{]}\PY{p}{)}
\end{Verbatim}
\end{tcolorbox}

    \begin{tcolorbox}[breakable, size=fbox, boxrule=1pt, pad at break*=1mm,colback=cellbackground, colframe=cellborder]
\prompt{In}{incolor}{ }{\boxspacing}
\begin{Verbatim}[commandchars=\\\{\}]
\PY{c+c1}{\PYZsh{} Objects detected on each frame}

\PY{n}{images} \PY{o}{=} \PY{p}{[}\PY{p}{]}

\PY{k}{for} \PY{n}{i} \PY{o+ow}{in} \PY{n}{x}\PY{p}{:}

  \PY{n}{input\PYZus{}tensor} \PY{o}{=} \PY{n}{tf}\PY{o}{.}\PY{n}{convert\PYZus{}to\PYZus{}tensor}\PY{p}{(}\PY{n}{i}\PY{p}{)}
  \PY{n}{input\PYZus{}tensor} \PY{o}{=} \PY{n}{input\PYZus{}tensor}\PY{p}{[}\PY{n}{tf}\PY{o}{.}\PY{n}{newaxis}\PY{p}{,} \PY{o}{.}\PY{o}{.}\PY{o}{.}\PY{p}{]}

  \PY{n}{detections} \PY{o}{=} \PY{n}{detect\PYZus{}fn}\PY{p}{(}\PY{n}{input\PYZus{}tensor}\PY{p}{)}

  \PY{n}{num\PYZus{}detections} \PY{o}{=} \PY{n+nb}{int}\PY{p}{(}\PY{n}{detections}\PY{o}{.}\PY{n}{pop}\PY{p}{(}\PY{l+s+s1}{\PYZsq{}}\PY{l+s+s1}{num\PYZus{}detections}\PY{l+s+s1}{\PYZsq{}}\PY{p}{)}\PY{p}{)}

  \PY{n}{detections} \PY{o}{=} \PY{p}{\PYZob{}}\PY{n}{key}\PY{p}{:} \PY{n}{value}\PY{p}{[}\PY{l+m+mi}{0}\PY{p}{,} \PY{p}{:}\PY{n}{num\PYZus{}detections}\PY{p}{]}\PY{o}{.}\PY{n}{numpy}\PY{p}{(}\PY{p}{)}
                \PY{k}{for} \PY{n}{key}\PY{p}{,} \PY{n}{value} \PY{o+ow}{in} \PY{n}{detections}\PY{o}{.}\PY{n}{items}\PY{p}{(}\PY{p}{)}\PY{p}{\PYZcb{}}
  \PY{n}{detections}\PY{p}{[}\PY{l+s+s1}{\PYZsq{}}\PY{l+s+s1}{num\PYZus{}detections}\PY{l+s+s1}{\PYZsq{}}\PY{p}{]} \PY{o}{=} \PY{n}{num\PYZus{}detections}

  \PY{n}{detections}\PY{p}{[}\PY{l+s+s1}{\PYZsq{}}\PY{l+s+s1}{detection\PYZus{}classes}\PY{l+s+s1}{\PYZsq{}}\PY{p}{]} \PY{o}{=} \PY{n}{detections}\PY{p}{[}\PY{l+s+s1}{\PYZsq{}}\PY{l+s+s1}{detection\PYZus{}classes}\PY{l+s+s1}{\PYZsq{}}\PY{p}{]}\PY{o}{.}\PY{n}{astype}\PY{p}{(}\PY{n}{np}\PY{o}{.}\PY{n}{int64}\PY{p}{)}

  \PY{n}{image\PYZus{}np\PYZus{}with\PYZus{}detections} \PY{o}{=} \PY{n}{i}\PY{o}{.}\PY{n}{copy}\PY{p}{(}\PY{p}{)}

  \PY{n}{viz\PYZus{}utils}\PY{o}{.}\PY{n}{visualize\PYZus{}boxes\PYZus{}and\PYZus{}labels\PYZus{}on\PYZus{}image\PYZus{}array}\PY{p}{(}
      \PY{n}{image\PYZus{}np\PYZus{}with\PYZus{}detections}\PY{p}{,}
      \PY{n}{detections}\PY{p}{[}\PY{l+s+s1}{\PYZsq{}}\PY{l+s+s1}{detection\PYZus{}boxes}\PY{l+s+s1}{\PYZsq{}}\PY{p}{]}\PY{p}{,}
      \PY{n}{detections}\PY{p}{[}\PY{l+s+s1}{\PYZsq{}}\PY{l+s+s1}{detection\PYZus{}classes}\PY{l+s+s1}{\PYZsq{}}\PY{p}{]}\PY{p}{,}
      \PY{n}{detections}\PY{p}{[}\PY{l+s+s1}{\PYZsq{}}\PY{l+s+s1}{detection\PYZus{}scores}\PY{l+s+s1}{\PYZsq{}}\PY{p}{]}\PY{p}{,}
      \PY{n}{category\PYZus{}index}\PY{p}{,}
      \PY{n}{use\PYZus{}normalized\PYZus{}coordinates} \PY{o}{=} \PY{k+kc}{True}\PY{p}{,}
      \PY{n}{max\PYZus{}boxes\PYZus{}to\PYZus{}draw} \PY{o}{=} \PY{l+m+mi}{200}\PY{p}{,}
      \PY{n}{min\PYZus{}score\PYZus{}thresh} \PY{o}{=} \PY{l+m+mf}{.4}\PY{p}{,} \PY{c+c1}{\PYZsh{} Adjust this value to set the minimum probability boxes to be classified as True}
      \PY{n}{agnostic\PYZus{}mode}\PY{o}{=}\PY{k+kc}{False}\PY{p}{)}
  
  \PY{n}{images}\PY{o}{.}\PY{n}{append}\PY{p}{(}\PY{n}{image\PYZus{}np\PYZus{}with\PYZus{}detections}\PY{p}{)}
  
\PY{c+c1}{\PYZsh{}\PYZsh{} The following lines visualize the detections on the image}

  \PY{c+c1}{\PYZsh{} \PYZpc{}matplotlib inline}
  \PY{c+c1}{\PYZsh{} plt.figure(figsize=IMAGE\PYZus{}SIZE, dpi=200)}
  \PY{c+c1}{\PYZsh{} plt.axis(\PYZdq{}off\PYZdq{})}
  \PY{c+c1}{\PYZsh{} plt.imshow(image\PYZus{}np\PYZus{}with\PYZus{}detections)}
  \PY{c+c1}{\PYZsh{} plt.show()}
\end{Verbatim}
\end{tcolorbox}

    \begin{tcolorbox}[breakable, size=fbox, boxrule=1pt, pad at break*=1mm,colback=cellbackground, colframe=cellborder]
\prompt{In}{incolor}{ }{\boxspacing}
\begin{Verbatim}[commandchars=\\\{\}]
\PY{c+c1}{\PYZsh{} Each frame is exported to \PYZsq{}images\PYZus{}new\PYZsq{} directory }

\PY{k+kn}{from} \PY{n+nn}{PIL} \PY{k+kn}{import} \PY{n}{Image}
\PY{k+kn}{import} \PY{n+nn}{numpy} \PY{k}{as} \PY{n+nn}{np}

\PY{n}{w} \PY{o}{=} \PY{l+m+mi}{0}

\PY{k}{for} \PY{n}{i} \PY{o+ow}{in} \PY{n}{images}\PY{p}{:}
  \PY{n}{img} \PY{o}{=} \PY{n}{Image}\PY{o}{.}\PY{n}{fromarray}\PY{p}{(}\PY{n}{i}\PY{p}{,} \PY{l+s+s2}{\PYZdq{}}\PY{l+s+s2}{RGB}\PY{l+s+s2}{\PYZdq{}}\PY{p}{)}
  \PY{n}{w} \PY{o}{=} \PY{n}{w} \PY{o}{+} \PY{l+m+mi}{1}
  \PY{n}{img}\PY{o}{.}\PY{n}{save}\PY{p}{(}\PY{l+s+s1}{\PYZsq{}}\PY{l+s+s1}{/content/gdrive/MyDrive/Project/images\PYZus{}new/}\PY{l+s+s1}{\PYZsq{}} \PY{o}{+} \PY{n+nb}{str}\PY{p}{(}\PY{n}{w}\PY{p}{)} \PY{o}{+} \PY{l+s+s1}{\PYZsq{}}\PY{l+s+s1}{.png}\PY{l+s+s1}{\PYZsq{}}\PY{p}{)}
\end{Verbatim}
\end{tcolorbox}

    \begin{tcolorbox}[breakable, size=fbox, boxrule=1pt, pad at break*=1mm,colback=cellbackground, colframe=cellborder]
\prompt{In}{incolor}{ }{\boxspacing}
\begin{Verbatim}[commandchars=\\\{\}]
\PY{c+c1}{\PYZsh{}  Frames from \PYZsq{}images\PYZus{}new\PYZsq{} directory are converted into a video}

\PY{k+kn}{import} \PY{n+nn}{numpy} \PY{k}{as} \PY{n+nn}{np}
\PY{k+kn}{import} \PY{n+nn}{glob}
 
\PY{n}{img\PYZus{}array} \PY{o}{=} \PY{p}{[}\PY{p}{]}
\PY{k}{for} \PY{n}{filename} \PY{o+ow}{in} \PY{n}{glob}\PY{o}{.}\PY{n}{glob}\PY{p}{(}\PY{l+s+s1}{\PYZsq{}}\PY{l+s+s1}{/content/gdrive/MyDrive/Project/images\PYZus{}new/*.png}\PY{l+s+s1}{\PYZsq{}}\PY{p}{)}\PY{p}{:}
    \PY{n}{img} \PY{o}{=} \PY{n}{cv2}\PY{o}{.}\PY{n}{imread}\PY{p}{(}\PY{n}{filename}\PY{p}{)}
    \PY{n}{height}\PY{p}{,} \PY{n}{width}\PY{p}{,} \PY{n}{layers} \PY{o}{=} \PY{n}{img}\PY{o}{.}\PY{n}{shape}
    \PY{n}{size} \PY{o}{=} \PY{p}{(}\PY{n}{width}\PY{p}{,}\PY{n}{height}\PY{p}{)}
    \PY{n}{img\PYZus{}array}\PY{o}{.}\PY{n}{append}\PY{p}{(}\PY{n}{img}\PY{p}{)}
 
 
\PY{n}{out} \PY{o}{=} \PY{n}{cv2}\PY{o}{.}\PY{n}{VideoWriter}\PY{p}{(}\PY{l+s+s1}{\PYZsq{}}\PY{l+s+s1}{/content/gdrive/MyDrive/Project/images\PYZus{}new/project\PYZus{}1.avi}\PY{l+s+s1}{\PYZsq{}}\PY{p}{,}\PY{n}{cv2}\PY{o}{.}\PY{n}{VideoWriter\PYZus{}fourcc}\PY{p}{(}\PY{o}{*}\PY{l+s+s1}{\PYZsq{}}\PY{l+s+s1}{DIVX}\PY{l+s+s1}{\PYZsq{}}\PY{p}{)}\PY{p}{,} \PY{l+m+mi}{15}\PY{p}{,} \PY{n}{size}\PY{p}{)}
 
\PY{k}{for} \PY{n}{i} \PY{o+ow}{in} \PY{n+nb}{range}\PY{p}{(}\PY{n+nb}{len}\PY{p}{(}\PY{n}{img\PYZus{}array}\PY{p}{)}\PY{p}{)}\PY{p}{:}
    \PY{n}{out}\PY{o}{.}\PY{n}{write}\PY{p}{(}\PY{n}{img\PYZus{}array}\PY{p}{[}\PY{n}{i}\PY{p}{]}\PY{p}{)}
\PY{n}{out}\PY{o}{.}\PY{n}{release}\PY{p}{(}\PY{p}{)}
\end{Verbatim}
\end{tcolorbox}


    % Add a bibliography block to the postdoc
    
    
    
\end{document}
